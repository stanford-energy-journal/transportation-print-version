\documentclass{papertex}
\usepackage[utf8]{inputenc}

%% Document begins here
\begin{document}

\begin{news}{2}{Driving into an Age of Increasing Oil Freedom}{Deron Lovaas}{}{}

We are living in exciting times when it comes to the nation’s oil and energy 
dependence. You could call this era the age of increasing oil freedom. From 
the mid-1980s to the mid-2000s the U.S. used and imported more oil every year, 
but since the mid-2000s that trend has reversed, with lower oil consumption 
and lower oil imports every year. To be clear, I’m not claiming that we are 
at or near energy independence, a clarion call that has been sounded by U.S. 
Presidents since Nixon, and I will not be discussing oil production, which is 
only one half of the equation. I am simply making an observation that is 
obvious from examination of recent trends in U.S. oil consumption, 
specifically in our transportation sector which accounts for the lion’s share. 
Whether and how this trend of lower oil consumption continues is up to our 
national and state government.

First, let’s examine how efficiently we’re using this resource in our 
world-leading fleet of cars and trucks. In the early 2000s oil prices began 
rising at an alarming rate. I still recall a comment made by one of NRDC’s 
communications experts as we worked on a study: the scenario in which the 
price of oil would rise to \$100 per barrel just didn’t “pass the laugh test.” 
Just three short years later the price of oil broke the \$140 mark. This is a 
reminder that ~\$100 per barrel prices are a relatively new normal, and this 
backdrop has added a boost to policy drivers described below.

\image{images/oilfree}{Credits: Nayu Kim}

This oil-price rocket boosted the chances that new policies could be 
implemented to reduce the nation’s oil dependence, and that’s in fact what 
happened, especially from 2007 on. In 2007, one short year after stunning the 
world by declaring that “America is addicted to oil” in his State of the Union 
address, President Bush signed the Energy Independence and Security Act into 
law. This law gave Bush’s successors the tools to dramatically raise the 
fuel-economy-performance bar for our fleet of cars and trucks. And that is 
what President Obama has done, in a series of historic rulemakings such as 
those in 2012 and 2013. These policies are driving record jumps in fuel 
economy and help explain the plateauing of long-term oil consumption 
projections.

Beyond vehicle efficiency, another discontinuity is putting the squeeze on oil 
consumption: flattening trendlines in the number of miles driven per capita, 
called vehicle-miles-of-travel (VMT). As I wrote recently, this has spurred 
analysts to reduce their projections of VMT growth, which forms the baseline 
for energy use and pollution estimates for the transportation sector. By 2030, 
projected demand drops more than one-fifth compared to the 2008 baseline, so 
one-trillion vehicle miles vanish every year! Conventional wisdom said this 
would turn around once the economy recovered, but as the trend continues more 
analysts think change may be structural and therefore lasting. Even 
bureaucratically calcified state highway agencies are changing their 
projections.

This is all very good news for energy security and the environment. However, 
while federal policy drives the first set of efficiency trends by raising the 
standards for fuel economy, it has yet to contribute to the second set by 
shaping the nation’s transportation infrastructure so people can drive less.

Transportation law relies heavily on state authority to plan and invest in 
infrastructure. However, federal infrastructure spending can be hugely 
influential because it accounts for about one-fifth of the nation’s 
transportation investments every year, and since spending federal dollars 
requires local, state, and/or private matches it leverages a lot more spending 
than that.

This is why the reauthorization of the national transportation law is a focus 
of policymakers, advocates, industry and state and local officials every few 
years. The most recent law, Moving Ahead for Progress in the 21st Century 
(MAP-21), was enacted in 2012 and it’s sadly a lackluster statute when it 
comes to saving energy and the environment.

Transportation law made a huge leap forward two decades ago in the Intermodal 
Surface Transportation Efficiency Act (ISTEA), which as Robert Puentes at 
Brookings wrote recently remains relevant to transportation even if its 
promise of more balance in authority between metropolitan regions and states 
and between modes (e.g. vehicle, bus, rail) remains largely unfulfilled. This 
bill was followed by two sequels (TEA-21, SAFETEA-LU) that preserved the basic 
architecture of the original and shared its “TEA” namesake.

MAP-21 includes a notable set of steps forward in its planning sections, 
specifically a performance management process with measurements being designed 
right now by federal rulemaking and guidance (view the implementation schedule 
here). However, tying performance to funding – which seems logical – is facing 
resistance from recalcitrant state highway bureaucrats.

And this is exactly where the next battle lines are drawn for advancing 
transportation energy policy beyond vehicle efficiency. State agencies must 
change colors – challenging though it may be – to become leaders and not 
laggards in the energy security race. Vehicle performance standards are doing 
their part to improve fuel economy performance of our fleet, yet states remain 
poor partners with metropolitan areas where the vast majority of us live 
(especially in the suburbs, like yours truly).

We may be entering a new era, however, for three reasons. First, the lasting 
moderation of VMT growth as described above. Whatever the factors driving it – 
demographic (millenials forsaking the car culture, as another article in this 
issue describes, economic, environmental, etc. – it’s something to which even 
state highway agencies must adjust by re-balancing investment portfolios.

Second, as the saying attributed to Churchill goes, “We have run out of money, 
time to start thinking.” The federal gas tax hasn’t budged since 1993, and in 
the interim inflation and additional needs have clobbered the spending power 
of this revenue. We are funding our federal program increasingly on the 
nation’s already overburdened credit card. States and local jurisdictions are 
filling in some of the gap, but the revenue picture for them is a threadbare 
patchwork quilt. With the last transportation bill, this revenue squeeze sadly 
made Congress and states more risk averse, not less, so policy changes tended 
to be regressive or nonexistent. We need more, not less, progressive 
innovation from our policymakers. We have no alternative but to start 
thinking.

And last, thanks to the fiscal crunches and ensuing scrutiny of public 
expenditures, we may be at a turning point for state transportation policy. 
The best evidence of this for me is a new evaluation of one of the largest 
state highway agencies, the California Department of Transportation 
(Caltrans), by the State Smart Transportation Initiative at the University of 
Wisconsin. This remarkable report, commissioned and then embraced by Caltrans’ 
parent agency, the California State Transit Agency (CalSTA), recommends that 
the agency reform itself from top-to-bottom. It must become, among other 
things, better at collaborating with and supporting metropolitan area planning 
and investment, and more balanced in terms of its focus on modes of 
transportation (i.e., less focused on sprawl-inducing, oil-guzzling highways).

That’s the kind of balance ISTEA promised for transportation, and in the wake 
of historic federal fuel-efficiency policymaking it is high time Caltrans and 
other state highway agencies get serious about fulfilling that promise so the 
nation can continue up the road to increased freedom from oil.


\subsubsection*{}

\emph{Deron Lovaas is Director of State/Federal Policy \& Practice for the 
Natural Resources Defense Council (NRDC)’s Urban Solutions Program. He has 
worked at the intersection of transportation, community development and energy 
policy for two decades. Before joining NRDC, he worked at other environmental 
organizations including the National Wildlife Federation, the Sierra Club and 
Maryland’s state Department of Environment. He has a degree from the University 
of Virginia.}

\end{news}

\end{document}
