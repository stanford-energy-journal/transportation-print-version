\documentclass{papertex}

\begin{document}

\begin{news}{2}{The Simple Power of the Bicycle}{Leah Shahum}{}{}

Let's be honest: The facts and figures about the threat of global warming can be
overwhelming, so much so that it is common for people to feel paralyzed by
inaction when they hear climate change warnings. They may be left to wonder
``Well, what could I possibly do about such a tremendous challenge?"

Well, I am fortunate to witness every day what individuals can do to make a real
difference. I work in the slightly obscure, though fast-growing, field of
bicycle advocacy, helping thousands of people each year lessen their carbon
footprints in San Francisco by trading 4-wheel trips for 2 wheels. In fact, our
work at the San Francisco Bicycle Coalition has boosted bicycling by 96\%
between 2006 and 2013. Meanwhile, the number of auto trips declined more than
8\% between 2006 and 2011.

And San Francisco is not alone in decreasing energy consumption by boosting
biking. Cities across the nation, including Chicago, Washington DC,
Philadelphia, St. Louis, Cleveland, and our friends across the San Francisco
Bay, Oakland, have significantly increased bicycling (and walking and transit)
trips in the past few years while lowering–or at least holding steady–private
auto trips. Even in “car is king” California, the percentage of trips taken by
biking, walking, and taking transit has doubled over the past decade from 11\%
in 2000 to nearly 23\% in 2012.

This shift is particularly important because transportation is the largest
single source of air pollution in the United States. According to the Union of
Concerned Scientists, fine particulate matter alone, much of which comes from
transportation related emissions, is responsible for up to 30,000 premature
deaths each year. Yet transportation is one of the areas that we, as
individuals, have the most ability to control. A whopping 60\% of carbon
emissions generated by transportation in this country originates from cars and
light trucks (the remainder mainly from heavy-duty vehicles and airplanes). The
average person who bikes five miles to work, five days a week, avoids 2,000
miles of driving a year – the equivalent of 100 gallons of gasoline saved and
2,000 pound of CO2 emissions avoided. This equates to saving 5\% of the average
American’s carbon footprint. This means that regular people like you and me have
the greatest potential to turn this problem around.

Simply put, too many people are burning too much fuel in single-occupancy
vehicles. Fortunately, a surprisingly straightforward, inexpensive, and low-tech
solution to this problem is right under our noses or, more likely, stored in our
garages: \textbf{ \textit {THE BICYCLE}}

The simple power of the bicycle is already apparent in cities across the world.
The citizens of thriving places such as Copenhagen and Amsterdam (both similar
in population and density to San Francisco) are already making between 30\% and
60\% of their trips on bicycles. These are the models that my bicycle advocacy
colleagues and I look toward for inspiration. We know it can be done because it
is already being done.

Notably, people bicycling in places like Copenhagen and Amsterdam are not doing
so because someone convinced them that biking is good for the environment or
because they care more deeply than Americans do about lowering their carbon
footprint. No, the 38\% of Amsterdammers riding (57\% in the city’s core) are
doing so largely because they recognize that biking is easier, faster, cheaper,
and more enjoyable than driving a car.

This has not occurred by chance. It stems from decision-makers consciously
investing in great bicycling environments. For example, in Amsterdam, officials
have designated 250,000 bike parking spaces (located closer to train stations
than car parking), built 250 miles of physically separated bikeways, and slowed
speeds on other roads to ensure people of all ages and speeds are comfortable
(especially those on the outer ends of the age spectrum: children and their
grandparents).

Americans are no different than our friends in northern Europe when it comes to
making basic lifestyle choices. How we decide to get to work, bring our kids to
school, or move around for errands or recreation is largely based on what’s most
convenient and expedient for us personally, not some grand environmental
motivation. If it’s easier and faster to drive, we usually do. If transit is
most convenient, we take the bus. If biking proves speediest and most enjoyable,
then we’ll pedal.

Unfortunately for our environment (as well as our pocketbooks and general
health), American communities have largely been built to prioritize automobiles,
making it a challenge to see bicycling, transit, or walking trips as the most
attractive options in many places.

This is where the work of bicycle advocates can be most effective. Our focus is
on improving the policies, plans, and investments needed to make communities
bicycle-friendly so that more people have the option of biking for more of their
trips. This means ensuring cities build comfortable, safe bikeways that connect
neighborhoods, job centers, commercial corridors, and transit. It also means
implementing policies that encourage bicycling and ensuring citizens have access
to secure bike parking and public transit. Given that half of all trips in U.S.
cities are 3 miles or less, simple, low-cost investments in bicycling can make a
real difference in affecting many people’s choice to ride.

An impressive number of American cities — including San Francisco and other
communities that have seen an increase in the number of people riding — are
ramping up their investments and building better bicycling infrastructure.

And when it comes to bicycling, nothing could be truer than the adage: when you
build it, they will come. The most impactful investment that city leaders can
make in encouraging bicycling trips is to create designated, and preferably
physically separated, spaces on the roads for the growing number of people
riding. These must be complete, interconnected networks of comfortable bikeways
that get people where they need to go. Creating new bikeways draws new riders —
which means fewer cars on the road, less competition for limited parking spaces,
and more space on public transit for those who need it—not to mention less
energy consumption, less congestion, and a healthier community.

Other key changes that have boosted San Francisco’s bicycle ridership include
making transit fully accessible to bikes, particularly regional connections such
as BART and Caltrain, requiring all new commercial buildings to create secure
bike parking, and compelling existing office spaces to accommodate tenant’s
bicycles. Robust encouragement programs such as annual Bike to Work Day, Safe
Routes to School, and bike safety classes are inspiring more people to ride, and
bikeshare programs, which are spreading like wildfire across American cities,
offer more people the option to bike for short trips.

In fact, since the Bay Area bikeshare program was launched last August, more
than 120,000 trips have been taken on the 700 bikes in San Francisco and South
Bay cities; many of these people are new to bicycling or are coming back after a
hiatus since their college days or even childhood. Denser cities like New York
City and Chicago are seeing even higher usage of their bikeshare systems. Many
of these bike rides are replacing short car trips, which have the greatest
negative impact on the environment and energy use due to cold engine starts.

Why is all of this significant? Because, worldwide, people are moving to cities
at an eye-popping rate — by the middle of the 21st century, the urban population
will almost double, increasing to 6.4 billion in 2050. We must focus on the most
efficient and effective ways to move many more people in far more crowded areas.
The suburban model of two cars per family will not work because we simply do not
have enough space to sustain it. And even the most hard-working public transit
systems can only be expanded to carry so many more people.

We must look to successful bicycling cities such as Amsterdam and Copenhagen as
models of smart, energy-efficient transportation planning. This is important not
only in America, but also in developing countries, where the current trends of
increasing driving and decreasing bike ridership are particularly frightening.

Fortunately for all of us, investing in great bicycling communities is also one
of the most affordable and quick ways we can shift transportation patterns in
our cities to be energy-efficient.

In the end, it should be as easy as riding a bike!

\subsubsection*{}

\emph{Leah Shahum has served as Executive Director of the San Francisco Bicycle
Coalition since 2002. For four years prior, she was the SFBC’s Program Director.
Before coming to the SFBC, Leah worked as a journalist for a local newspaper and
a national magazine. She graduated from Duke University in North Carolina with a
degree in political science and a focus in women’s studies. Leah most recently
served on the Board of Directors of the SF Municipal Transportation Agency. She
previously served for five years on the 19-member Board of Directors of the
Golden Gate Bridge, Highway, and Transportation District. She is also a San
Francisco representative to the state Democratic Party}

\end{news}

\end{document}
