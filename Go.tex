\documentclass[10pt]{papertex}
\usepackage[utf8]{inputenc}

%% Colour for heading
\definecolor{color}{rgb}{0.7,0.2,0.2}
%% Edition - Issue 4 ... . Regular font, larger than in template.
\edition{
    \raisebox{-2pt}
    {\fontsize{6mm}{7mm}\usefont{T1}{bch}{b}{n}{Issue 4: Sustainable Transportation}
    }}
%% Document begins here
\begin{document}

%% Hack to make the Date/Time show as Month YYYY.
\makeatletter
\renewcommand{\papertex@headDateTime}{
    \raisebox{-5pt}
    {\fontsize{5mm}{6mm}\usefont{T1}{bch}{b}{n}{April 2014}}
    }
\makeatother
%% Footer hack
\fancyfoot[C]{\copyright Stanford Energy Journal \the\year}
%% "Logo" - Stanford Energy Journal in large font.
\mylogo{{\fontsize{12mm}{14mm} \usefont{T1}{bch}{b}{n} 
    \noindent\textcolor{color}{Stanford Energy Journal}}%
}

\begin{news}{2}{Millennial Mobility Preferences: Good News for Sustainable Transportation}{Lucian Go}{}{}

It no longer takes a transportation planner to see the shift occurring in the 
travel habits of the millennial generation, which is the largest and most 
diverse in American history. Born between 1982 and 2003, millennials are 
driving less, walking and biking more, and gravitating towards urban areas 
that facilitate shared-use transportation services. In doing so, this 
generation is beginning to chip away at a decades-old constant in American 
transportation: the reign of the personal automobile. What factors are 
contributing to this change in behavior—and what kind of benefits could it 
present for the environment?

As a number of recent studies have found, one of the biggest ways in which 
millennials differ from previous generations is through a growing preference 
for alternatives to solo driving, which is resulting in an unprecedented 
decrease in per capita driving, referred to as vehicle miles travelled (VMT). 
In 2009, the average American between 16 and 34 drove a stunning 23\% less 
annually than they did in 2001—a decrease of 2400 miles per capita.

\subsubsection*{The Millennial Mindset}

The American Public Transportation Association’s recent report, entitled 
\emph{Millennials \& Mobility: Understanding the Millennial Mindset}, found 
through polling that millennials are increasingly multi-modal, with 69 percent 
of respondents using multiple transportation options to reach a destination a 
few times a week or more. The report found that on average, millennials use 
three different transportation options on a typical trip (including walking), 
mainly due to the lower cost and convenience of traveling that way. Polling 
conducted by Zipcar has shown that the increased availability of on-demand 
mobility services, such as carsharing and ridesharing, has a greater impact 
on the decisions of millennials than on those of older generations. 
Additionally, millennials appear more concerned about the amount of driving 
they do than previous generations; 44 percent of millennials say they are 
making a conscious effort to reduce it by using other modes of transportation, 
as opposed to 30 percent of those ages 45-54 and just 26 percent of those 
ages 55 or older.

By shifting from personal car use towards multi-modal public-transit trips, 
millennials are taking a more pragmatic approach towards travel, increasingly 
weighing the fact that cars, long considered a symbol of freedom and 
convenience, are not always the easiest way to get around, and certainly not 
the cheapest.

Although millennials are decreasing their driving the most dramatically, they 
are part of a larger, unprecedented change in American behavior that started 
a decade ago. After a near-constant increase in annual per capita driving for 
sixty years, VMT appears to have peaked in 2004—and has now been decreasing 
for eight straight years. Research by U.S. PIRG found that a higher than 
average decline in driving did not seem to correlate with higher than average 
unemployment, suggesting that this behavior cannot be brushed off as a 
temporary side effect of the recession.

\image{images/lyft}{Carsharing Service - Lyft, Photo: Ya-Ko}

\subsubsection*{Carsharing and Environmental Benefits}

Of the transportation alternatives that millennials are turning to in lieu of 
personal car ownership, carsharing programs are one of the fastest-growing, 
and with the transportation sector accounting for over a quarter of U.S. 
greenhouse gas emissions, one of the most intriguing in terms of potential 
long-term environmental benefits.

A study conducted by the University of California estimated that every 
carsharing vehicle removes 9 to 13 private cars from the road. With an 
estimated fleet of 15,600 shared vehicles in North America as of January 
2013, that would mean the removal of up to 200,000 private vehicles from the 
road—or about half the registered autos in San Francisco County. This fleet 
of shared cars is quickly growing in size, having increased by over 23 percent 
from January 2012 to January 2013.

According to another study conducted by the University of California in 2011, 
carsharing was estimated to have decreased greenhouse gas emissions in North 
America by between 158,000 and 224,000 tons per year, based on 2008 data. 
With North American carsharing membership surpassing a million members in 
2013, the potential for significant emissions reductions has grown rapidly. 
Access to a shared car enables users to converge to a low-mileage, shared-use 
lifestyle. For many users that do not own a personal vehicle, the report makes 
the observation that access to this lifestyle may actually increase their 
greenhouse gas emissions. However, this is more than offset by car-holding 
households who give up their cars, who tend to decrease their emissions 
significantly.

As these carsharing services continue to proliferate, they are becoming a 
more and more popular option, even for car-owning households. Additionally, 
increased use of transportation modes like transit, walking and biking are 
growing hand-in-hand with services like carsharing as part of a ``new 
mobility'' lifestyle. As U.S. PIRG’s recent report A New Way to Go 
highlights, cities and planning agencies can do a lot to further this trend 
away from car dependence and the pollution and congestion associated with it. 
By modernizing regulations to embrace multi-modality, transportation planners 
can help to spread more, and better, transportation choices to Americans, 
whether they live in cities or suburbs.

\subsubsection*{}

\emph{Lucian Go works in the Transportation and Urban Solutions programs at 
the Natural Resources Defense Council (NRDC). Besides exploring and writing 
about millennial transportation habits and technology-enabled mobility 
services as they relate to sustainability, Lucian’s work involves advocacy 
for California’s greenhouse gas emissions reduction efforts, particularly 
with regards to transportation. Lucian is an incoming master’s student at the 
Yale School of Forestry and Environmental Studies, and holds a BA from the 
University of California, Berkeley.}

\end{news}

\end{document}
