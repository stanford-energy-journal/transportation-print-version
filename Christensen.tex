\documentclass{papertex}
\usepackage[utf8]{inputenc}

\begin{document}

\begin{news}{2}{Renewable Fuel Standard: Driving Change or Maintaining Status 
    Quo?}{Adam Christensen}{}{}

It is said that you should never discuss religion and politics in polite 
company, but sometimes I feel like this saying should be never discuss 
religion, politics and biofuels in polite company. In the best situations, 
when I say that I work on biofuels policy, I get an interested nod and a 
polite question about fuels made from algae. In the other extreme I get a 
look that borders on a scowl and then a comment about how biofuels are a 
waste of government money and that the fuels made from bio sources (whatever 
those might be) are sub-standard. That’s quite a range of reactions, and one 
that has motivated me to dig into the details of this controversy.

To start a rational discussion it is first necessary to say that there are many 
different types of biofuel. Biofuel is just a general name for any type of fuel 
that happens to be made from biomass. Let’s start by splitting the term biofuel 
in half and discuss the bio side and the fuel side separately. The term fuel 
certainly has its legal definitions, but generally it is any chemical molecule 
that can be burned in an internal combustion engine. In the United States that 
means any chemical that can approximate gasoline or diesel. The vast majority 
of biofuel in the US is ethanol, an alcohol that can be mixed with gasoline, 
and the most common diesel replacements are fatty acid methyl esters 
(biodiesels). The bio side of the fuel equation is much more complicated as 
these two fuels (alcohols and biodiesels) can be made from a variety of biomass 
sources and newer scientific processes are opening the doors to even more 
options. However, to even define the term biomass (sometimes referred to as 
renewable biomass) is to invite controversy related to overall sustainability 
of the finished fuel due to land use change and carbon versus non-carbon 
emissions. The US government doesn’t even have a single definition of 
renewable biomass. Different parts of the government use different definitions 
all with slightly different specifications. These subtleties can mean the 
difference between receiving a tax credit, being eligible for a grant, or 
being able to participate in the Renewable Fuel Standard (RFS) … or not.

Policies such as the Renewable Fuel Standard (RFS) are attempting to regulate 
the embedded emissions within a biofuel by establishing a complex mandate to 
use four different types of biofuels (cellulosic, advanced, biomass-based 
diesel, and renewable) each with their own, different, greenhouse gas (GHG) 
emission threshold. However, the emissions benefits of using biofuels are 
captured upstream of the tailpipe and there are no standardized chemical tests 
that allow someone to measure lifecycle emissions. There is a wide range of 
biofuel and those with lower associated lifecycle emissions, such as cellulosic 
fuel, should be rewarded. However, investors are less likely to take on risk 
by investing in cellulosic fuel when the incentives to do so are ambiguous.

The RFS is the largest driver of biofuel use in the United States, and one of 
the world’s most aggressive biofuel policies. The RFS mandates the use of a 
certain amount of biofuel in our transportation infrastructure. There is no 
other option for regulated parties unless they want to pay heavy fines for 
non-compliance. To comply with the RFS, a regulated party must own a certain 
number of compliance certificates, called Renewable Identification Numbers 
(RINs) for each of the four previously mentioned fuel categories. RINs are 
generated by biofuel producers and typically sold along with the biofuel to 
obligated parties. RINs were seen as a market tool to ease the burden of 
compliance because they can be traded separately from the biofuel with which 
they were bought, and to provide another revenue stream for cellulosic fuel 
producers.

The problem is that there is very little cellulosic biofuel production. As a 
result the EPA is in the awkward position of having to mandate the use of a 
fuel that essentially does not exist. To ease this pressure the EPA has the 
authority to reduce the required volume of cellulosic fuel as prescribed by 
Congress in the RFS. This process of revising the volume requirements is 
referred to as the EPA’s waiver authority. The EPA is supposed to set the 
revised volume mandate at the beginning of each calendar year, but this is an 
inherently uncertain process. To date, EPA has mandated cellulosic fuel use 
every year, but that fuel ultimately was not available, which required 
additional Cellulosic Waiver Credits (CWCs), or waivers.

These waivers can be purchased instead of RINs directly from the EPA under 
certain circumstances. The problem with CWCs is that their price was set by 
Congress and is pegged to the price of gasoline. Its value per gallon is 
calculated as the maximum of either \$3.00 minus the price of gasoline or 
\$0.25. Waiver credits have been inexpensive in the past, and under gasoline 
price scenarios predicted by the Energy Information Administration, these 
credits will only continue to decrease in price and undermine the purpose of 
the RFS. At first cellulosic fuel will likely come at a premium, and that 
premium will be included in the price of RINs that producers would sell to 
obligated parties. Obligated parties would likely pay whatever that RIN price 
is because they must show compliance with the RFS, but because CWCs will be 
inexpensive, obligated parties will always choose a CWC over an RIN, stripping 
away a vital revenue stream for the new cellulosic fuel industry.

The subtitle of the Energy Independence and Security Act that created the RFS 
was called “Energy Security through Increased Production of Biofuels.” Congress 
has made it clear that the intent was to address security issues, but Congress 
was also interested in promoting the cellulosic biofuels industry in order to 
obtain environmental benefits. The existence of the waiver credit together 
with the waiver authority undoes much of the work to incentivize such dramatic 
market change. As such, compliance with the RFS has essentially been a game 
between first generation biofuels (corn ethanol and biodiesel) and imported 
sugarcane ethanol from Brazil and cellulosic fuels have been unable to gain a 
market share. It is difficult to envision how the RFS, in its current form, 
can truly drive change towards cellulosic fuels, and the EPA has an uphill 
battle to climb in order to instill market confidence.

\subsubsection*{}

\emph{Adam Christensen is a National Science Foundation Science, Engineering 
and Education for Sustainability (SEES) Fellow and works with Johns Hopkins 
as a postdoctoral researcher. His research focuses on evaluating biofuel tax 
credits against key sustainability criteria and exploring opportunities that 
could make these credits more sustainable in the future. Adam graduated with 
his Ph.D. in Mechanical Engineering in 2009 from the Georgia Institute of 
Technology. Following graduation, he worked on Capitol Hill as an American 
Society of Mechanical Engineering Congressional Fellow and also was the Staff 
Engineer for the Appliance Standards Awareness Project, an energy efficiency 
advocacy group.}

\end{news}

\end{document}
