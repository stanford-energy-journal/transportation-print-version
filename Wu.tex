\documentclass{papertex}
\usepackage{amsmath}

\begin{document}

\begin{news}{2}{Fuels for Fresh Air}{Diane Wu, in conversation with Mark Jacobson}{}{}

As you read this, chances are the air that you’re breathing contains thousands of particles per cubic centimeter. The air and atmosphere that surrounds us is a dynamic and largely invisible landscape that is shaped by many factors, including emissions from vehicles. I spoke with Professor Mark Jacobson, an expert at describing and modeling this invisible world.
\\
\\
\textbf{How does the transportation sector contribute to greenhouse gas 
    emissions in the United States today?}
\\
\\
About one third of climate-relevant emissions are from transportation: 26\% 
from vehicle exhaust and 7-8\% from the upstream production of fuels.

Emissions from transportation also contribute to air pollution. In the United 
States, about 50,000 to 100,000 people die prematurely each year from air 
pollution. Transportation emissions are responsible for about a third of those 
deaths, or 17,000 to 33,000 deaths each year. Emissions from transportation 
are particularly potent because they occur at street level, where people are 
much more likely to breathe them in than the emissions from power plants. The 
intake fraction, or likelihood that pollutants will be inhaled, is on the 
order of10 to 30 times higher for vehicles than for power plant pollution.
\\
\\
\textbf{Which pollutants emitted from vehicles are particularly concerning for 
    public health?}
\\
\\
Particulate matter causes 80-90\% of mortality from air pollution, and ozone 
is much of the rest. There are other chemicals that contribute to health 
conditions, especially carcinogens such as benzene, butadiene, acetaldehyde, 
and formaldehyde.
\\
\\
\textbf{How do emissions from diesel compare with those from gasoline?}
\\
\\
Diesel vehicles produce a significant portion of transportation air pollution. 
Diesel is a primary source of particulate matter. For both global warming and 
human health, it’s not just the overall amount of emissions that matters, it’s 
also the type of particles, i.e. their volume, size, and number. The particles 
created in diesel combustion are very small – around 0.1 microns, well below 
the 2.5 micron diameter cutoff for particles that are small enough to filter 
into your lungs and cause health problems.

In 2008, the California Air Resources Board passed a Truck and Bus Rule that 
requires heavy diesel vehicles to be fitted with particulate traps. These 
traps reduce about 90\% of particulates coming from diesel vehicles, but this 
is still not enough to be lower than gasoline. Moreover, these traps have 
several problems in terms of higher fuel use requirements and more ozone 
formation.

The type of emissions from diesel fuel are also much more potent for the 
atmosphere. Most of diesel emissions are black carbon, which has a much higher 
global warming potential (GWP) than CO2. Over a one year period, black carbon 
has more than a million times the GWP per unit mass than CO2, and over 100 
years it has on the order of 1000 times more GWP. When you account for the 
GWP of the black carbon from diesel, even when you have a particle trap, the 
benefit from using a diesel vehicle disappears.

But the comparison shouldn’t only be between diesel and gasoline. There are 
other alternatives to clean up the transportation sector, such as electric 
vehicles and hydrogen fuel cell vehicles.
\\
\\
\textbf{How do the emissions from electric vehicles and hydrogen fuel cell 
    vehicles compare with those from internal combustion engines?}
\\
\\
Electric vehicles have no emissions. The only emissions from hydrogen fuel 
cell vehicles are water vapor – about the same amount of water vapor that an 
internal combustion vehicle produces – and possibly some leaked hydrogen. 
Hydrogen gas doesn’t have any impact on health in the short term, and in the 
long term the impact is trivial. Nothing else. There’s no particulate matter, 
no nitrous oxides, no sulfur oxides, no organic gases, no carbon monoxide.
\\
\\
\textbf{What about emissions from upstream production of electricity or 
    hydrogen?}
\\
\\
Producing electricity from solar or wind and generating hydrogen through 
electrolysis creates no emissions, except what is necessary to manufacture 
the wind turbines or solar cells. But even if you power your electric vehicles 
with the background electric power grid in the US, you still would reduce the 
total CO2 emissions by about 30\% compared to burning gasoline. Furthermore, 
you move the emissions from the streets, where a larger population can be 
exposed to the pollutants, and emit them instead at power plants. Ideally, of 
course, you would use wind or solar.

Steam reforming of natural gas is the most common way of producing hydrogen 
today. This process emits carbon dioxide and trace amounts of other 
chemicals – a very small amount compared to what vehicle exhausts puts out. 
But because it requires the production of methane, it still contributes to 
the GWP of the overall fuel cycle.
\\
\\
\textbf{What are some promising alternatives to hydrocarbon fuels for air 
    transportation?}
\\
\\
Hydrogen produced from electrolysis and cooled to a liquid would be a clean 
alternative fuel. For example, the space shuttle ran on cryogenic hydrogen to 
get it lifted to space. In the 1980s the Russians modified an aircraft that 
ran on cryogenic hydrogen. More recently, in 2003, the European Commission 
published a report on hydrogen powered aircraft. It can be done. You have 
larger airplanes, but they weigh less, so the total drag is similar. It is 
technically feasible. Even economically I think it would be similar to current 
aircraft, but there’s an upfront cost and an infrastructure change that’s 
required. The key will be to create policies to make these changes, since 
airplane manufacturers have no reason to change the status quo.
\\
\\
\textbf{Do you anticipate any unintended consequences in switching to electric 
    vehicles?}
\\
\\
No, I don’t. I’ve driven an electric vehicle for four and a half years, and I 
haven’t personally experienced any unintended consequences. There have 
actually been many benefits that I didn’t expect. It’s quiet and it 
accelerates so nicely. There are fewer moving parts, so fewer things can 
break down. It’s been a great experience for me to drive an electric car.

\subsubsection*{}

\emph{Prof. Mark Z. Jacobson is the Director of the Atmosphere / Energy Program 
within the Department of Civil and Environmental Engineering at Stanford 
University. The main goal of Prof. Jacobson’s research is to understand 
better severe atmospheric problems, such as air pollution and global warming, 
and develop and analyze large-scale clean-renewable energy solutions to them.}
\\
\\
\emph{Diane Wu is a graduate student in Chemistry at Stanford University. 
Working with Prof. Jennifer Dionne and Prof. Alberto Salleo, she researches 
materials that convert light from lower to higher energies, which have nascent 
applications in solar energy conversion and bioimaging. Diane is also a 
producer at Green Grid Radio, a student-run environmental storytelling 
podcast.}

\end{news}

\end{document}
