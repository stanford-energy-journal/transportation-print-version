\documentclass{papertex}

\begin{document}

\begin{news}{2}{Autonomous Mobility-On-Demand -- A Solution for Sustainable Urban Personal Mobility}{Rick Zhang}{}{}

According to the United Nations, the world population living in urban areas 
will double by 2050, with most of the urbanization occurring in the developing 
regions of Asia and Africa. Meanwhile, private car ownership continues to 
skyrocket, causing traffic congestion and pollution in the increasingly dense 
urban environment with limited available land for road expansion and parking 
spaces. All of this points towards the reality that private automobiles, which 
have served all of our personal mobility needs in the past century, are an 
unsustainable solution for the future of personal urban mobility.

One of the most promising approaches to solving the personal urban mobility 
problem is through one-way car sharing with light electric vehicles, also 
known as mobility-on-demand (MOD) systems. Car sharing has the potential to 
increase vehicle utilization rates, which are currently below 10\% for 
private automobiles, and to promote sustainable urban land use, since fewer 
cars require less parking space. In the past few years, MOD services have 
begun operation in dozens of cities throughout North America and Europe, with 
notable examples including CAR2GO and Autolib’ Paris.

\subsubsection*{\textbf{Where Automonous Cars Rule}}

However, because some destinations in a city are more popular than others, 
customer trips to popular destinations will deplete cars from certain areas of 
the city, causing imbalance in the system. This imbalance strongly affects the 
availability of vehicles throughout the city and negatively impacts the 
convenience that was promised by this type of system. Proposed solutions to 
this problem include offering price incentives to customers, encouraging ride 
sharing/splitting, or hiring a team of drivers to rebalance the vehicles. 
However, incentivizing ride sharing adds inconvenience for customers and 
hiring rebalancing drivers increases costs for the service operator. This 
leads to increased prices for customers and less spending on maintenance and 
customer service, which are crucial for customer satisfaction. Indeed, a quick 
glance at the reviews for CAR2GO on Yelp reveals many of these underlying 
problems.

Very recently, a transformational technology is emerging where on-demand 
mobility is provided by driverless cars. The technology of robotic vehicles 
has progressed rapidly since the DARPA Grand Challenge in 2005, and robotic 
vehicles designed specifically for personal mobility are already being produced 
and marketed (for example, the Google car, Induct NAVIA, and General Motors’ 
EN-V vehicle). Robotic vehicles hold great promise for MOD systems because 
they can rebalance themselves, thus eliminating the problem of system imbalance 
at its core. In addition, autonomous vehicles can relieve commuters from the 
burdens of driving, enable system-wide coordination, and potentially increase 
safety. Finally, through system-level coordination, autonomous vehicles can 
use existing road infrastructure more efficiently by following other 
autonomous vehicles more closely and help alleviate congestion by routing 
vehicles away from heavily congested areas. All of these benefits can 
translate into long-term environmental sustainability and potential cost 
savings for customers without sacrificing the convenience of personal 
mobility.

\subsubsection*{\textbf{Reduced Environmental Costs}}

The environmental benefits of autonomous MOD systems result from the fact that 
car sharing yields high utilization rates and that vehicles are restricted to 
operate in a limited urban setting. An autonomous MOD system in a city will 
consist of mostly a single type of vehicle, which can be mass produced and 
maintained cheaply and efficiently by leveraging economies of scale. The high 
utilization rates of these vehicles will result in a shorter vehicle life 
span, allowing new technologies to be quickly deployed and integrated into 
the system. Sustainable recycling practices can be implemented to minimize 
the impact of manufacturing new vehicles. Purely electric vehicles are very 
suitable for autonomous MOD systems because vehicles are restricted to 
operate within the boundaries of a city. They will be able to charge their 
batteries when not serving customers, their speeds will be limited by what is 
practical in urban driving (around 20 mph), and charging constraints can be 
taken into account through system coordination to guarantee that vehicles have 
enough charge to carry passengers to their destinations. The relatively slow 
speeds of the vehicles, along with additional safety due to autonomous driving, 
means that the vehicles can be lighter, further enhancing their energy 
efficiency and sustainability.

For example, the MIT CityCar, the original vehicle proposed for MOD systems, 
weighs only 450 kg (40\% the mass of a mid-sized sedan). If we made all urban 
trips in light electric vehicles such as the CityCar rather than privately 
owned gasoline-powered sedans, we would cut our energy expenditure by almost 
50\%, after taking into account additional trips made for system rebalancing. 
Depending on the energy infrastructure of the city, the CO2 emissions can 
potentially be reduced to zero. Furthermore, the batteries in the electric 
vehicles provide a large bank of energy storage and can sell electricity back 
into the power grid to mitigate peak demand, which is beneficial to renewable 
but intermittent energy sources such as solar and wind.

\subsubsection*{\textbf{How Does it Work?}}

The practical benefits of autonomous MOD systems can lead to their widespread 
implementation in the next 5 to 10 years. Currently in the busiest cities in 
the US, traffic congestion adds an additional 60 hours per commuter per year 
in travel time for commuters. If commuters didn’t have to be burdened with 
driving, this wasted time on the road can be turned into productive work, 
which can be converted into an average cost saving of over \$800 per person 
per year! Perhaps most importantly, when operated correctly, an autonomous 
MOD system can in fact increase convenience compared to privately owned 
vehicles. Customers will be able to call a vehicle from their smartphones 
just prior to leaving their home, and by the time they step outside, the 
vehicle will be waiting for them. Customers can be productive while inside 
the vehicle and upon arrival at their destinations, would not need to worry 
about finding parking.

Through intelligent coordination techniques, autonomous MOD systems has the 
potential to drastically reduce the number of vehicles we need. A recent study 
conducted jointly between Stanford and the Singapore-MIT Alliance for Research 
and Technology (SMART) found that if we replaced all forms of transportation 
in Singapore with a single autonomous MOD system, we would need only need 
300,000 vehicles, or one third the current number of cars in Singapore. 
Finally, robotic vehicles also hold the promise to increase safety by 
eliminating human factors, which contribute to 95\% of current traffic 
accidents.

\subsubsection*{\textbf{The Way Forward}}

Despite all the advantages of autonomous MOD systems, there are two key 
challenges that must be addressed before their widespread implementation. 
Firstly, even though autonomous driving technology has progressed a long way 
in the past decade, there is virtually an infinite number of possible traffic 
scenarios and weather conditions that an autonomous vehicle might find itself 
in. The robustness of the algorithms controlling autonomous vehicles have not 
been validated in these scenarios and therefore the absolute safety of the 
vehicles cannot be guaranteed. The problem arises particularly when autonomous 
vehicles interact with pedestrians and other human-driven vehicles. To tackle 
this problem, we must learn more about how humans interact with autonomous 
vehicles through small scale deployment and extensive testing.

Secondly, system-wide coordination of autonomous vehicles is essential to 
realizing the full financial benefits and convenience of autonomous MOD 
systems. One key criticism of autonomous MOD systems is that they may in fact 
increase traffic congestion due to the additional (empty) vehicles that are on 
the road en route to their next pickup. While it is true that more vehicles 
will be on the road for the same number of passenger trips, preliminary studies 
of simulations on road networks conducted at Stanford have shown that empty 
vehicles redistributing themselves throughout the city will typically travel 
along less-congested roads, and are unlikely to contribute to the maximum 
congestion in the city. However, this is highly dependent on how intelligently 
vehicles are routed throughout the city.

Research into autonomous MOD systems are still in its infancy, and while they 
hold great promise to provide sustainable urban personal mobility in the near 
future, several technological advancements are needed to fully realize their 
potential. The success of such a system is inherently tied to the commuter’s 
willingness to adopt this technology because, ultimately, the MOD model 
represents a paradigm shift away from private ownership towards collective 
sharing and public ownership of our transportation infrastructure for a more 
sustainable future.

\end{news}

\end{document}
