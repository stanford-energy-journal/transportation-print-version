\documentclass{papertex}
\usepackage[utf8]{inputenc}


%% Document begins here
\begin{document}

\begin{news}{2}{Sustainable Air Travel for a Carbon-Free Future}{Michael Colonno and Juan J. Alonso}{}{}

Commercial aviation is one of the fastest growing sources of greenhouse gas 
emissions and yet a critical component of the global economic infrastructure. 
A recent report, co-authored by the U.S. Department of Transportation, 
forecasts global carbon dioxide emissions due to commercial aviation of 1.5 
billion tons per year by 2025, considerably worse than previous predictions of 
the International Panel on Climate Change. By comparison, the entire European 
Union, some 457 million people, currently emits about 3.1 billion tons of 
carbon dioxide annually. The same report found that growth of carbon dioxide 
emissions on this scale will outstrip any gains made by improved technology 
by a wide margin, ensuring that commercial aviation is an even larger 
contributor to global warming by 2025 than previously thought.

Despite the considerable greenhouse gas contribution, the industry is 
economically driven by fuel efficiency. With approximately 40\% of annual 
operating costs coming from fuel, airlines constantly push for more efficient 
aircraft and engines. Growth is so rapid that the projections above exist in 
spite of a 70\% increase in industry-wide fuel efficiency over the last four 
decades. A projected 1.4-3x growth in the number of flights by 2025 signals 
that fundamental technological change is needed to curb greenhouse gas 
emissions in a significant way without severe economic restrictions. The only 
path to long-term reductions in greenhouse gas emissions is to power commercial 
aircraft with a greenhouse-free fuel. Many alternatives for sustainable 
aviation fuels with significant reductions in life-cycle carbon costs are 
being investigated, (sustainable crops, biomass, etc.) and they are likely to 
play a role in the reduction of the carbon footprint of commercial aviation. 
An alternative is to leverage electricity and / or hydrogen as fuel sources 
to contribute to a sustainable future for the industry. In addition, both 
noise pollution and the total cost of operation of the commercial fleet may 
be considerably reduced, resulting in both economic and additional 
environmental drivers.

Propulsion systems for commercial aircraft, while experiencing considerable 
improvements in efficiency through better engineering, materials, coatings, 
and changes to the thermodynamic cycle, have not fundamentally changed in 
decades. Hydrocarbon fuel is burned with oxygen from the atmosphere to drive 
a thermodynamic cycle and produce thrust. This – a turbofan engine – can be 
thought of a small gas turbine power plant under each wing, the output from 
which turns a fan to push the aircraft through the air. Fundamentally 
rethinking the way aircraft are propelled through the atmosphere presents 
numerous challenges but also some advantages not available to traditional 
engines. The key to this flexibility is the potential use of electric current 
to distribute power to the propulsion systems rather than fuel (chemical 
energy). This, in turn, allows for the use of motors to drive propulsion. 
Motors are reliable, quiet, and highly efficient devices that are already 
manufactured in a wide range of sizes. Efficiency is a critical differentiator: 
a state-of-the-art turbofan, bound by thermodynamics, utilizes less than 40\% 
of the chemical energy in its fuel while a well-designed motor can operate 
well above 90\% efficiency.

In order to leverage this efficiency gain, the greatest of carbon-free 
aviation’s technical challenges must be faced: storing a large amount of 
energy. A Boeing 737-800, widely used by Southwest Airlines among others, 
requires more than 11 MW of continuous power when cruising – equal to that 
of about 8,400 average households in the United States. Using the same units 
of a residential electricity bill, typical jet fuel stores more than 9.8 kWh 
of raw energy (heat when combined with oxygen) in every liter or about 12 kWh 
in every kilogram. If all of the chemical energy were converted to electrical 
energy, about 3.2 liters would power an average household for a day. It is a 
refined energy resource that is a stable liquid under ambient conditions, 
meaning it can be easily transported and pumped. In order to utilize 
electrical power, we require a storage medium that either holds electrical 
energy directly or can be converted into electrical energy continuously. The 
former describes batteries and the latter describes fuel cells.

Batteries are the most tempting energy storage medium; electric cars have seen 
a resurgence, powered by advances in lithium-ion batteries. They are a simple, 
safe, and well-established technology. Despite their success in automotive 
applications, batteries have to become a lot lighter before they can power an 
airline-sized aircraft. Even the most advanced, readily-available battery 
systems of today have a peak energy density of around 0.20 kWh per kg, some 
1/60th of jet fuel. Lithium-air batteries, under active research, may break 
the 1.0 kWh per kg level over the next several years but still fall an order 
of magnitude under hydrocarbon fuels. Without a transformational breakthrough 
in battery technology, a battery-powered airliner with sufficient range and 
speed isn’t getting off the ground.

Can we retain the high energy density of chemical energy storage while 
utilizing the efficiency of electrical power? The answer is yes, though there 
is no shortage of engineering challenges. Hydrogen, which must be stored in 
the form of either a highly-compressed gas or extremely cold liquid, can be 
reacted with oxygen from the atmosphere in a fuel cell to produce electric 
current. Fuel cells are also a well-established, robust technology that has 
been demonstrated successfully in automotive and some smaller aviation systems. 
Hydrogen has terrific energy density when considering weight – three times 
that of gasoline – but takes up a lot more volume, even in a liquid state. 
That additional volume increases the size of the aircraft and adds to its 
weight. Despite this difficulty, hydrogen fuel cell technology provides a 
viable near-term solution for carbon-free aviation.

Assuming the technology challenges can be solved, an economic case would 
still have to be made to the companies operating the aircraft. Motors, fuel 
cells, and batteries are all much simpler systems than turbofan engines and 
it is reasonable to assume that maintenance costs – a major expense for 
airlines – would be reduced. The most profound change in the business model, 
however, is the transition from an energy derived from a natural resource 
(traditional jet fuel) to an energy currency derived from the grid. While 
hydrogen can be purchased relatively inexpensively, almost all hydrogen 
produced today is derived from natural gas, simply releasing the carbon 
dioxide on the ground instead of in the air. A carbon-free solution requires 
hydrogen made through a simple process called electrolysis (a fuel cell in 
reverse) in which water is split, via electric current, into its constituents: 
hydrogen and oxygen. Hence, hydrogen serves merely as a temporary storage 
mechanism for grid power. Provided the necessary power is available from the 
grid, airlines or suppliers can manufacture the energy required to fly when 
and where it is needed. The aircraft, ultimately, are as green as the power 
grid supporting them.

While hydrogen-powered aircraft include numerous environmental advantages, 
including zero in-flight greenhouse gas emissions and noise reduction, the 
environmental impacts of water vapor emissions need to be carefully considered. 
While less harmful and better regulated by natural processes than carbon 
dioxide, water vapor is a greenhouse gas and contrails from jets play a 
significant (and complex) role in atmospheric thermal balance. These impacts, 
as well as the very new operational and maintenance procedures associated 
with carbon-free aircraft, will need to be explored as they are, hopefully, 
introduced into the next generation of our commercial fleet.

\subsubsection*{}

\emph{Michael Colonno is an Engineering Research Associate at the Aerospace 
Design Lab in the Department of Aeronautics and Astronautics at Stanford 
University. He works with Prof. Juan Alonso on advanced aircraft design and 
green aviation. Michael holds an M.S. and a PhD in Aeronautics and Astronautics 
at Stanford. Prior to Stanford, he also was the Director of the Aerodynamics 
Group at Space Exploration Technologies.}

\end{news}

\end{document}
