\documentclass[10.5pt]{papertex}
\usepackage[utf8]{inputenc}

%% Colour for heading
\definecolor{color}{rgb}{0.7,0.2,0.2}
%% Edition - Issue 4 ... . Regular font, larger than in template.
\edition{
    \raisebox{-2pt}
    {\fontsize{6mm}{7mm}\usefont{T1}{bch}{b}{n}{Issue 4: Sustainable Transportation}
    }}
%% Document begins here
\begin{document}

%% Hack to make the Date/Time show as Month YYYY.
\makeatletter
\renewcommand{\papertex@headDateTime}{
    \raisebox{-5pt}
    {\fontsize{5mm}{6mm}\usefont{T1}{bch}{b}{n}{April 2014}}
    }
\makeatother
%% Footer hack
\fancyfoot[C]{\copyright Stanford Energy Journal \the\year}
%% "Logo" - Stanford Energy Journal in large font.
\mylogo{{\fontsize{12mm}{14mm} \usefont{T1}{bch}{b}{n} 
    \noindent\textcolor{color}{Stanford Energy Journal}}%
}

\begin{news}{2}{A Path Towards More Sustainable Personal Mobility}{
    Sven Beiker}{}{}

In the United States, light duty vehicles travel about three trillion 
passenger miles every year. That means that annually we collectively travel a 
distance more than thirty thousand times the distance from the Earth to the 
Sun! While we appreciate the convenience of personal automobile travel, we 
also know that our behavior has negative impacts, especially on the 
environment.

Every year, we burn 140 billion gallons of gasoline in our automobiles, 
contributing to climate change and causing a measurable decrease in air 
quality. In many metropolitan areas the pollution from traffic creates visible 
smog, which is especially a problem in developing countries. Finally, oil 
resources are slowly being depleted and the control of remaining, valuable 
oil supplies has been a persistent source of international conflict.

If the effects of driving personal vehicles are so bad, why do we continue 
using them? The problem is that the ability to move around freely and 
independently in a personal vehicle is a fundamental value of our society. So 
what can we do? There are three possible solutions that could mitigate the 
problems described above before the negative impacts lead to irreversible 
damage to our ecosystem, and ultimately to the exhaustion of our natural 
resources. These solutions, which likely will be implemented over the next 
few decades, are electrification, automation, and transportation sharing.

Electrification means that conventional powertrain components are supported 
or completely replaced by electric motors and batteries. Different levels of 
electrification are possible, such as hybrid electric, plug-in hybrid, battery 
electric, or fuel cell vehicles. These are promising solutions to the climate, 
pollution, and resource problems that I have described above. For example, an 
electrified powertrain can be powered by many different sources of electricity. 
This would allow cars to be powered mainly by renewable energy rather than 
only by fossil fuels, as is the case with the combustion engines that power 
most of our cars today. In addition, an electric powertrain is much more 
efficient than a traditional combustion engine in everyday stop-and-go traffic, 
so electrification of cars brings both lowered emissions and increased energy 
efficiency.

However, electric vehicles are not a perfect solution. Electricity can come 
from many sources and some of them, like coal, are no better for the 
environment than burning gasoline. Also, while it is possible to cover 
everyday driving with an electric motor, long-distance driving might still 
require combustion engines for a number of reasons. One of the main reasons 
is the limited range of electric vehicles. In contrast, the range of a 
gasoline powered vehicle is virtually unlimited because even if the tank is 
empty, one can refuel within minutes basically anywhere. Additionally, the 
combustion engine – if reasonably sized –is actually not a bad solution for 
long-distance travel at relatively constant speeds. Therefore, plug-in hybrids 
and range extended electric vehicles should and will probably be a preferred 
solution for personal automobiles for the next 10-20 years, until hydrogen 
fuel cell vehicles and the infrastructure required for their widespread 
adoption are figured out. Once the infrastructure is in place fuel cell 
vehicles will require no supplementary combustion engine to extend their range 
and will not degrade over many charge-discharge cycles like a battery does. 
They probably will, when mass-produced, require less weight and space for 
energy storage than an electric vehicle. All of these advantages will hopefully 
allow fuel cell vehicles to have an even lower impact on climate, air quality, 
and resources than battery powered vehicles.

The next solution to our personal transportation issue is automation wherein a 
computer system takes over driving, at least in part, from the human. The main 
motivation for this is improved safety because in 95\% of all accidents human 
error is at least a contributing factor. However, centralized and anticipatory 
vehicle control can also help lower the fuel consumption of the vehicle for 
several reasons. First of all, computer controlled vehicles could be spaced 
closer together and therefore take advantage of decreased air resistance, 
increasing fuel economy. This would also increase the efficiency of traffic 
overall, meaning that all cars would spend less time on the road. Today each 
of us is stuck in traffic for about 36 hrs. every year and this time could be 
significantly reduced through automation of personal vehicles. Secondly, humans 
often have slightly erratic driving patterns instead of precisely adapting 
their speed to match the overall traffic flow. This means that a vehicle 
consumes less gas when driven on cruise control than when the driver tries to 
maintain a constant speed manually. A central traffic coordination center would 
take this to another level by arranging for a smooth and energy optimal 
traffic flow which would greatly reduce fuel consumption, consequently cutting 
energy use and reducing emission of pollutants.

For complete automation a communication link to connect as many vehicles as 
possible to a centralized control would be necessary, and fortunately this 
seems feasible within the next 10-15 years. In the meantime, sensors on the 
vehicle, such as radars, lasers, and cameras, will allow individual vehicles 
to be automated. These sensors will help control the vehicle smoothly and 
efficiently based on the traffic pattern of the surrounding vehicles and 
improve efficiency on an individual basis.

A third way to make personal mobility more sustainable is transportation 
sharing, which will help to save energy and reduce emissions as well. 
Transportation sharing was possible but inconvenient before the widespread 
use of mobile, internet-connected technologies, but now there are several ways 
it can be used conveniently. One way is car sharing, which saves about 40\% of 
energy and emissions compared to an individually owned vehicle. This is 
because a driver becomes more aware of every trip if he has to make special 
arrangements before getting in the car. This suggests that driving is just too 
convenient. We need to consider whether making it less convenient is a 
direction that society as a whole is willing to accept.

However, other options for sharing transportation, such as carpooling, can 
also help. With a broad variety of devices, apps, and cloud services, the 
personal automobile will become much more integrated into our digital 
lifestyle and will therefore be easier to use more efficiently. For example, 
if a driver knew that someone in the same group of friends wanted to go to the 
same place at the same time, why would they not carpool? Hopefully carpooling 
will significantly increase vehicle occupancy from 1.4 people per vehicle today 
and thereby save energy and reduce emissions.

A final option for transportation sharing is making public transit easier to 
use. Integration of the automobile into a public transportation system would 
allow more efficient sharing of transportation resources and could be 
implemented via the mobile internet, which would allow transit information to 
be made available online in real-time. If travelers know they can get to their 
destination 20 minutes earlier by taking public transit and avoiding traffic 
rather than driving, and if public transit were easily accessible at park and 
ride stations, why would one not do it? Integrating the personal automobile 
into the mass transit system has the potential to greatly increase the 
efficiency of transportation overall.

We see that there are many solutions on hand to address the climate, air 
quality, and resource challenges that the use of personal vehicles creates. It 
is up to us how, as consumers, we use them, and probably even more how, as 
scientists and businesspeople, we implement them to create a much more 
efficient, sustainable, and also more enjoyable personal mobility system. I am 
very hopeful that we can do that.


\subsubsection*{}

\emph{Sven Beiker oversees the strategic planning, resource management, and 
project incubation for the Center for Automotive Research at Stanford (CARS). 
Besides managing industry relationships, he holds teaching positions at 
Stanford’s School of Engineering and the Graduate School of Business, and he 
conducts in electrical mobility at Stanford’s Precourt Energy Efficiency 
Center. Before coming to Stanford, Dr. Beiker worked with the BMW group for 
over 13 years where his responsibilities included technology scouting, 
innovation management, systems design, and series development. He holds 
several patents in chassis and powertrain systems.}

\end{news}

\end{document}
