\documentclass[10pt]{papertex}
\usepackage[utf8]{inputenc}

%% Colour for heading
\definecolor{color}{rgb}{0.7,0.2,0.2}
%% Edition - Issue 4 ... . Regular font, larger than in template.
\edition{
    \raisebox{-2pt}
    {\fontsize{6mm}{7mm}\usefont{T1}{bch}{b}{n}{Issue 4: Sustainable Transportation}
    }}
%% Document begins here
\begin{document}

%% Hack to make the Date/Time show as Month YYYY.
\makeatletter
\renewcommand{\papertex@headDateTime}{
    \raisebox{-5pt}
    {\fontsize{5mm}{6mm}\usefont{T1}{bch}{b}{n}{April 2014}}
    }
\makeatother
%% Footer hack
\fancyfoot[C]{\copyright Stanford Energy Journal \the\year}
%% "Logo" - Stanford Energy Journal in large font.
\mylogo{{\fontsize{12mm}{14mm} \usefont{T1}{bch}{b}{n} 
    \noindent\textcolor{color}{Stanford Energy Journal}}%
}


\begin{news}{2}{Sustainable Biofuels: The Leading Edge of Sustainable 
    Transport}{Gerard Ostheimer}{}{}

Transportation accounts for approximately 20\% of total global energy demand; 
therefore, de-carbonizing the transportation sector is fundamental to reducing 
global demand for fossil fuels. Fortunately, the international community is 
starting to act on the need for sustainable energy. UN Secretary General 
Ban-Ki Moon and World Bank President Jim Yong Kim have committed themselves 
and their institutions to the success of the Sustainable Energy For All 
initiative (SE4ALL), which seeks to 1) achieve universal energy access; 2) 
double the use of renewables; and 3) double the growth rate of energy 
efficiency by 2030. Admittedly, these goals are highly aspirational and 
arguably arbitrary. Nonetheless, SE4ALL is elevating the conversation to the 
highest levels of national governments and focusing the minds of world leaders 
on renewable energy. Striving for these targets will guide the global economy 
simultaneously towards sustainable development, climate change mitigation, and 
a future powered by low- and renewable-carbon technologies. Let us consider 
how to double the use of renewables in the transportation sector and the roles 
that renewable carbon can play moving forward.

The transportation sector represents a mix of technologies. Aviation, 
maritime, rail, heavy-duty vehicles, and light-duty vehicles for both public 
and private transportation all have inter-connected but diverse fueling 
requirements. In the fossil-fueled era, a single source is refined into 
diverse products that are specific to different transportation modes. In the 
renewable carbon era, fuels will be sourced by the inverse approach. Instead 
of diverse fuels from a single feedstock, diverse feedstocks will be 
channeled, in parallel, to their relevant demand. Sourcing different biomass 
feedstocks from diverse resources offers the potential for rural economic 
development across the globe, which stands in stark contrast to fossil fuel 
production.

Modes of personal transport are ever lighter and more efficient, and the 
market is beginning to offer both liquid and electric fuel options. 
Sustainable car fueling should account for local circumstances. In the bright 
sun of the American Southwest, electric vehicles are a natural fit, but in the 
mists of the Pacific Northwest, a sustainably sourced biofuel could be the way 
to go. Where biofuels will certainly be in demand for decades to come is the 
commercial transport of goods and people. Aviation, maritime, and heavy 
trucking require energy dense fuels, and hydrocarbons, despite our best wishes 
to the contrary, remain an efficient and convenient means of packaging chemical 
potential. Moving into the post-fossil fuel era, the key is to source 
hydrocarbons renewably. A diverse array of sustainably sourced biofuels will 
continue to fuel transportation in environmentally appropriate and 
technologically targeted ways.

Cognizant of the need for low-carbon energy solutions; the push to deploy 
sustainable biofuels is coming from both the public and private sectors. 
Multilateral organizations such as the International Energy Agency (IEA) and 
International Renewable Energy Agency (IRENA) see bioenergy and biofuels 
playing a considerable role in a de-carbonized future. In order to reduce CO2 
emissions by 50\% by 2050, the IEA forecasts biofuels rising to 27\% of world 
transportation fuels versus today’s 2\% . Serving as the SE4ALL renewable 
energy technology and policy “hub,” IRENA developed the Renewable Energy Map 
2030 that charts the way to doubling the amount of renewable energy. Biomass 
is expected to contribute 60\% of the increased use of renewables, with 
biofuels contributing 15\% of transportation fuel use . The private sector is 
leading the way. Boeing, Maersk, and Shell are among those driving technology 
development and deployment for transportation. KLM, SkyNRG, the Government of 
the Netherlands, and others recently announced the BioPort program to locally 
source aviation biofuels for Amsterdam’s Schipol international airport. It 
seems that the goal of doubling the use of renewables is not so far-fetched 
after all.

Given the competing demands on agricultural land, how will we source enough 
biomass such that renewable hydrocarbons can make a meaningful contribution to 
the energy mix? Due to some poorly conceived and wretchedly implemented biofuel 
projects in Africa and the unfortunate commodity price spike that occurred as 
U.S. biofuels production began to ramp up in 2007-2008, the conventional 
wisdom is that biofuels must compete with food production. Fortunately, 
biomass potential surveys from the U.S. DOE Billion Ton Study to various 
international evaluations (e.g. Dornburg et al., 2008) indicate supply 
sufficient to meet the IEA and IRENA scenarios. In addition, the Global 
Bioenergy Partnership, UN Food and Agriculture Organization, Inter-American 
Development Bank, and Roundtable on Sustainable Biomaterials have created 
tools for governments and businesses to guide the development of the bioenergy 
sector such that it is environmentally, socially, and economically sustainable. 
Biofuels should not be sourced from arid, food insecure regions, but where the 
land, water, and technical resources are available, bioenergy can be a 
powerful driver of investments in agriculture that drive increased 
agricultural production, improved resource management, and rural development.

As an example of the global potential for renewable hydrocarbons to help power 
sustainable development, consider the Addax Bioenergy Makeni sugar cane 
ethanol project. Addax Bioenergy has worked with the people of Sierra Leone 
to establish a bioenergy facility that dovetailed with local needs and 
potentials. They established a pivot irrigation system designed for local 
farmers to grow food crops between the circles of sugar cane. Building an 
ethanol biorefinery with a biomass-fired electricity co-generation plant, 
the Makeni project is now poised to produce 85 million liters of ethanol and 
supply 15 megawatts of electricity to the national grid (a 20\% boost in 
electricity production for the country). In addition, the project is certified 
sustainable by the Roundtable on Sustainable Biomaterials. Producing both 
liquid biofuels and bio-based power while creating 2,000 new jobs, the Makeni 
project is a promising example of a way towards achieving Sustainable Energy 
for All. The prospect of biofuels powering sustainable transport has been a 
roller coaster. Fortunately, a number of international institutions choose to 
focus on moving the bioeconomy forward and creating opportunities for biofuels 
to contribute to a low-carbon future.


\subsubsection*{}

\emph{Dr. Gerard Ostheimer is the Global Lead for Sustainable Bioenergy under 
the UN and World Bank initiative Sustainable Energy For All (SE4All). He 
promotes the development of well-functioning public-private partnerships that 
work towards achieving SE4All’s goals of increasing energy access and 
increasing the use of renewable energy. Previously, Dr. Ostheimer served as 
a Science Advisor for the Foreign Agriculture Service of the U.S. Department 
of Agriculture. He has a Ph.D. in molecular biology at the University of 
Oregon and did postdoctoral work in the systems biology of cancer at MIT.}

\end{news}

\end{document}
