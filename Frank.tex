\documentclass[10pt]{papertex}
\usepackage[utf8]{inputenc}


%% Colour for heading
\definecolor{color}{rgb}{0.7,0.2,0.2}
%% Edition - Issue 4 ... . Regular font, larger than in template.
\edition{
    \raisebox{-2pt}
    {\fontsize{6mm}{7mm}\usefont{T1}{bch}{b}{n}{Issue 4: Sustainable Transportation}
    }}
%% Document begins here
\begin{document}

%% Hack to make the Date/Time show as Month YYYY.
\makeatletter
\renewcommand{\papertex@headDateTime}{
    \raisebox{-5pt}
    {\fontsize{5mm}{6mm}\usefont{T1}{bch}{b}{n}{April 2014}}
    }
\makeatother
%% Footer hack
\fancyfoot[C]{\copyright Stanford Energy Journal \the\year}
%% "Logo" - Stanford Energy Journal in large font.
\mylogo{{\fontsize{12mm}{14mm} \usefont{T1}{bch}{b}{n} 
    \noindent\textcolor{color}{Stanford Energy Journal}}%
}

\begin{news}{2}{Rail Transportation and Energy Efficiency}{Richard Frank}{}{}

The California High-Speed Rail Authority (Authority) is the state agency 
responsible for planning, designing, building and operating the first 
high-speed rail system in the nation. As a member of the Authority’s Board of 
Directors, I’m pleased to provide an update on the project’s status and our 
plans for an energy-efficient transportation future.

First, the basics: California’s high-speed rail system will connect the major 
metropolitan areas of the state, create jobs, spur economic development, and 
enhance environmental quality – all while preserving agricultural and 
environmentally-sensitive lands. By 2029, the system will run from San 
Francisco to the Los Angeles basin in under three hours at speeds of over 
200 miles per hour. The system will ultimately extend to Sacramento and San 
Diego, totaling 800 miles with up to 24 stations.

We envision high-speed rail as an integral part of a broader state-wide rail 
modernization program that includes investment in existing urban, commuter, 
and intercity rail systems throughout the state.  Examples of these 
investments include the electrification of the Caltrain system, new rail cars 
for the Bay Area Rapid Transit (BART) system, the Regional Rail Connector in 
Los Angeles, and an upgrade of the Blue Line light rail system in San Diego. 
These new or upgraded services will connect existing transportation systems to 
high-speed rail, providing an integrated rail network that will serve as a 
viable alternative to vehicle and air travel.

California is large: our major economic and political centers are located 
hundreds of miles apart. As early as 1909, state leaders recognized this 
challenge and identified the need to develop major infrastructure projects to 
connect the regions of the state. Throughout the decades as the state’s 
population centers continued to grow, these crucial infrastructure investments 
served as the foundation for the state’s unprecedented economic growth and 
prosperity over the past century. However, these systems are rapidly aging, 
and increased demand caused by population growth and tourism require the 
development of new methods of transportation.

The Authority is committed to developing an environmentally- and 
economically-sustainable, long-term alternative to the current transportation 
systems in California, which will increase transportation efficiency, use 100 
percent renewable energy, and incorporate additional sustainability practices.

\subsubsection*{Rail as an Efficient, Clean Transportation Choice for
    California}

Although rail is one of the world’s oldest transportation methods, it continues 
to be one of the most energy efficient ways to move large numbers of people. 
Experience both in the U.S. and abroad illustrate that rail travel provides an 
efficient alternative for the traveler who wants to reach key, central 
destinations without the delay and hassle of air travel, and in a faster, 
less-polluting, safer and more efficient method than driving.

California’s high-speed rail system is designed to transport passengers 
throughout the state in the most energy efficient method possible. Starting in 
2022, the newly completed Initial Operating Section of high speed rail will 
provide non-stop service from Merced to the San Fernando Valley – with 
connections via cross-platform transfer to the Bay Area, Sacramento, and 
locations throughout Southern California. At that stage in operations, the 
system on any given day will require 0.40 gigawatt hours (GWh) to deliver 
11,000 passengers.  By 2029, non-stop service from the Bay Area to the Los 
Angeles area will move 528,000 passengers on approximately 2.30 GWh per day. 
To move the equivalent number of travelers by car would require nearly twice 
as much energy.

For operations, we will deploy a traction power system that will rely on two 
by twenty-five kilovolt alternating current Autotransformer Feed Systems for 
the main-line operation at each of the substations, located at 30-mile 
increments along the alignment.  The Authority has been hard at work with 
utility providers to study and provide this critical interconnection.

The system will also capture the energy from the braking and deceleration of 
the train. This energy, estimated to be 10 to 15 percent of the total power 
for the system, can be used in a number of ways, including: taken back into 
the breaking train to power electrical systems such as fans or lights; moved 
along the overhead catenary system to power other trains; or, returned to the 
electricity grid to bolster the state’s overall energy supply.

\subsubsection*{A Net-Zero Approach to Rail Operations}

Energy-efficiency and maximizing recovery of electricity are just parts of the 
Authority’s plans.  In 2008, the Authority’s Board of Directors adopted a 
policy goal to run operations with  100 percent renewable energy. Through 
subsequent planning and coordination with utility companies and regulatory 
agencies, the Authority has determined that the most effective, feasible way 
to achieve this goal is to procure or produce enough renewable energy to feed 
into the grid to offset the amount the train uses.

This net-zero approach means that renewable energy developers can find the 
most economical locations to develop and distribute energy to the grid.  In 
April 2013, through a formal call-to-industry process, the Authority determined 
that several companies have the capacity, and the strong interest, to provide 
renewable energy to the system.  In addition, the Authority is exploring solar 
or other renewable energy generation on high-speed rail canopies, roofs, and 
maintenance facilities as well as other structures.

In this fashion, California’s high speed rail system will be ahead of the 
clean transportation curve and leading by example.

\subsubsection*{Rail as Green as We Can Make It}

The Authority has also developed a sustainability policy in line with 
priorities set by Californians as part of their vote for the high-speed rail 
system (Proposition 1A) in 2008.

Strategies currently deployed by the Authority include items such as 
requirements  for recycling all steel and concrete; diversion of 75 percent 
of construction waste from landfills through reuse and recycling; use of new, 
low-emission construction equipment; and replacement of inefficient truck 
engines and irrigation pumps.  The Authority is also working with partners to 
implement an urban forestry program to offset greenhouse gas (GHG) emissions 
associated with construction.  This program will deliver additional benefits 
such as providing shade and recreation for communities within the train 
corridor. The Authority is also actively working with other state agencies to 
link the design-build contractor with biofuel suppliers so that a percentage of 
the fuel used for construction will be clean, renewable, or bio-diesel. 

So what’s the bottom line for high-speed rail as it relates to California’s 
energy future? Starting in 2022, during the system’s first year of operation, 
the Authority anticipates a ramp-up in usage as travelers begin to make the 
switch from driving or flying to taking high-speed rail. Every subsequent year, 
as the system is projected to increase riders, the Authority expects continued 
reductions in GHG emissions as well as criteria air pollutants.

Cumulatively, by 2030, the high speed rail system will divert between 4.3 
million and 8.2 million metric tons of carbon dioxide equivalent (mmtCO2e).  
That would be as if an entire 500-mile long lane of auto traffic were removed 
from California’s freeways.  By 2050, the system will divert at least 27 
million and possibly as much as 44.4 million metric tons of CO2e.  The 
documented performance of international high-speed rail systems, such as France 
and Spain, has shown that these numbers are reasonable.

These GHG reductions represent a major component of California’s pioneering, 
statewide efforts to steadily reduce the state’s aggregate GHG emissions, in 
partnership with our colleagues at the California Air Resources Board, Energy 
Commission and Public Utilities Commission.

As you can see, high-speed rail is a game-changing form of sustainable 
transportation. I can’t express how exciting it is to be part of this 
cutting-edge technology that will result in a brighter energy future for the 
state and serve as a model for the nation. While we still have a lot of work, 
we’re moving forward and will begin construction in the coming months.

\subsubsection*{}

\emph{Published as-submitted, with no edits or revisions, as requested by the 
California High Speed Rail Authority. Submitted Dec 04, 2013.}
\\
\\
\emph{Richard Frank is a Board Member for the California High-Speed Rail 
Authority and a Professor of Environmental Practice and Director of the 
California Environmental Law and Policy Center (CELPC) at the University of 
California at Davis School of Law.}

\end{news}

\end{document}
