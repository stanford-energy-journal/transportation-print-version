\documentclass{papertex}

\begin{document}

\begin{news}{2}{The Reports of My Death Have Been Greatly Exaggerated -- An Internal Combustion Engine’s Story}{John Fyffe}{}{}

\subsubsection*{Why aren’t IC engines a thing of the past?}

Internal combustion (IC) engines have been around for more than 150 years and 
helped usher in the industrial age.  With the rapid change of technology in 
the world today, one might suspect that IC engines will soon be a thing of 
the past, replaced by newer, better technologies.  While no one knows the 
answer yet, a strong case can still be made for the future of the IC engine.  
Research and development efforts are on-going to make modern IC engines more 
efficient, more environmentally friendly, more powerful, and capable of 
accepting a wider range of fuels.

\subsubsection*{Modern Engines}

Most Americans are familiar with the spark ignition (SI) engines found in most 
passenger cars in the US.  Spark ignition engines induct a mixture of fuel and 
air into the cylinder which is compressed and then ignited by a spark to create 
a flame that will sweep across the cylinder to burn all of the fuel.  This 
process converts the chemical energy in the fuel into sensible energy 
(increasing the temperature and pressure) that can be extracted in the form 
of work, usually to propel a car.  SI engines are still prominent because of 
their low cost, ease of use, and ability to meet a large dynamic range of 
loads (idle, city driving, highway driving, etc).

Compression ignition (CI) engines, commonly referred to as Diesel engines, 
are primarily used for heavy-duty transportation in the US, although they 
do make up a large portion of the light-duty market in Europe as well.  
Diesel engines differ from SI engines in that they induct only air into 
the cylinder and combustion occurs spontaneously, without the help of a 
spark – the mixture is said to autoignite – when fuel is injected directly 
into the compressed air.  Diesel engines can have higher compression ratios 
(CR), and therefore higher efficiencies, than SI engines because they avoid 
the problem of knock.  Knock occurs in SI engines when the flame from the 
spark compresses the unburned mixture that then explodes, creating a shockwave 
that causes engine damage (and the metallic “knocking” sound).  Another reason 
Diesel engines can have higher efficiencies is that they do not need to 
throttle the inlet air (which requires work from the engine) like SI engines.

Typical IC engines have peak efficiencies between 30 – 40\%.  However, 
engines are operated over a wide range of conditions – idling, accelerating, 
cruising, and stop-and-go – making the engine operate away from its ideal 
efficiency point. Under normal driving conditions the average efficiency of 
IC engines drops to approximately 20\%.  This is significantly lower than 
typical electricity-generating power plants that are between 35-55\% 
efficient, but IC engines in cars do operate under a more restrictive set 
of constraints, primarily the large range of operating conditions, small 
sizes, and low weight requirements.

Typical IC engines have peak efficiencies between 30 – 40\%.  However, 
engines are operated over a wide range of conditions – idling, accelerating, 
cruising, and stop-and-go – making the engine operate away from its ideal 
efficiency point. Under normal driving conditions the average efficiency of 
IC engines drops to approximately 20\%.  This is significantly lower than 
typical electricity-generating power plants that are between 35-55\% 
efficient, but IC engines in cars do operate under a more restrictive set 
of constraints, primarily the large range of operating conditions, small 
sizes, and low weight requirements.

Hybrid engine technology has developed rapidly in the past couple of decades, 
leading to improvements in efficiency and reduction in total emissions.  
Hybrids use battery powered electric motors to achieve the large dynamic 
driving range required while keeping the IC engine close to its most 
efficient operating point.  For example, when a driver is merging onto the 
highway and “steps on it”, the car requires more power than the IC engine 
produces at its peak efficiency, so the battery powered electric motor kicks 
in and helps accelerate the car.  But, if the driver ends up getting stuck 
in traffic and is only moving along at a few miles per hour, the engine is 
producing too much power at the maximum efficiency point.  Therefore, 
instead of reducing the power output of the engine (and reducing the 
efficiency at which it is producing power), the hybrid operates the electric 
motor in reverse, making the “motor” a generator.  Therefore, the engine can 
operate at peak efficiency, meet the driver’s demand of slogging through 
traffic, and store excess energy in the battery system on board to be used 
again when traffic clears up. Hence, hybrids can operate more efficiently 
than vehicles with only IC engines.

\subsubsection*{The Continued Evolution of IC Engines}

Moving forward, researchers are continuing to target efficiency and emissions 
improvements in IC engines.  Because they are ubiquitous, reliable, flexible 
devices, improving them further is an important endeavor.  As we try to 
improve IC engines, it’s important to understand where the inefficiencies 
arise.  Figure 1 shows the breakdown of where the incoming energy from fuel 
that is available for work output is going. A typical IC engine will lose 
20\% of its available work to the combustion process, 27\% is transferred 
out as heat, and 16\% is transferred out with the exhaust. This leaves only 
35\% of the energy available as work to move the vehicle.  Maximizing the work 
output therefore can be accomplished by reducing the heat transfer, 
extracting the available energy from the exhaust, and reducing the 
inefficiency associated with combustion.  Reducing losses due to heat transfer 
and exhaust are more easily tackled with current technologies.  The 
combustion loss is typically thought of as a sunk cost, however research 
done in the Advanced Energy Systems Lab (AESL) at Stanford shows that 
increased efficiencies can be achieved by combusting at higher energy states, 
referred to as the extreme states principle.  Our lab has shown that when the 
air/fuel mixture is combusted at higher and higher temperatures and 
pressures – for instance, in our experimental IC engine with a 100:1 
compression ratio – the air/fuel mixture becomes extremely energetic which 
reduces entropy generation during combustion, i.e. reducing the combustion 
losses.  While a 100:1 is not necessarily practical – the temperatures and 
pressures in such an engine would vastly exceed those in conventional IC 
engines – it demonstrates the extreme states principle and a method of 
reducing combustion losses.

There is work being done in many different areas to improve efficiencies, 
such as reducing weight and increasing compression ratios. One way to greatly 
affect an engine’s performance is to change the method of combustion 
(combustion strategy).  Two main combustion strategies currently being 
explored by research labs are: 1) gasoline direct injection (GDI) and 2) 
homogeneous charge compression ignition (HCCI).  Both of these techniques 
focus on changing the combustion strategy to improve efficiency and control 
emissions.

\subsubsection*{GDI}

GDI is a modified version of an SI strategy where a spark is used to ignite 
the mixture, but instead of injecting fuel into air before it enters the 
cylinder, gasoline is injected in-cylinder.  The primary benefit of GDI is 
the ability to control how much fuel is injected into the cylinder and where 
in the cylinder the fuel is located.  This means throttling, or passing the 
intake air through an obstruction to reduce the amount of air inducted into 
the cylinder, which requires work and reduces efficiency, can be avoided. 
It can also reduce NOx emissions and heat transfer losses by reducing peak 
combustion temperatures.  However, a current challenge with GDI is keeping 
the soot emissions below emissions standards.   Current GDI researchers are 
focusing on the best way to utilize the flexibility of the in-cylinder 
injections to produce high-efficiency and low emission IC engines.

\subsubsection*{HCCI}

HCCI is best thought of as a combination of an SI and Diesel engine.  HCCI 
engines induct (i.e. “breathe in”) a fuel/air mixture like SI engines.  
However, instead of a spark to start the combustion process, the fuel/air 
mixture is compressed until it autoignites.  HCCI engines are being pursued 
because they can go to higher compression ratios like Diesel engines, which 
increases efficiency, but have lower soot emissions than Diesels because 
the air and fuel are premixed.  Additionally, like GDI engines, HCCI engines 
have lower combustion temperatures and are therefore less prone to NOx 
emissions.  The current challenge is to develop an HCCI engine with the 
dynamic range of conventional IC engines.  The HCCI process is much more 
difficult to control compared to timing a spark (SI) or timing fuel injection 
(Diesel) because combustion occurs by autoignition.  Autoignition is governed 
by chemical kinetics, and therefore the fuel/air mixture’s temperature and 
pressure are what defines when combustion occurs.   Thus, HCCI engines have 
trouble at start up, higher engine speeds, and at higher power outputs. 
Current research is focused on solving these problems

\subsubsection*{Moving Forward}

With IC engine research moving forward to create more efficient engines with 
lower emissions, and with the development of economical and robust batteries, 
the transportation industry could see some phenomenal power plants put into 
cars in the near future.  Other areas of research that extend into the engine 
combustion community at large include PPCI / PCCI (partially premixed 
compression ignition / premixed charge compression ignition), multi-fuel 
engines, sootless diesels, biofuels/alternative fuels, hybridization, advanced 
boosting, and heat recovery or bottoming cycles.  Although the internal 
combustion engine has been around for more than a century, the engine 
community is ready to tackle the next millennium’s challenges by pushing the 
engineering envelope to create the IC engines of tomorrow.

\subsubsection*{}

\emph{John Fyffe is a Ph.D. student in Mechanical Engineering (ME) at Stanford University and works in the Advanced Energy Systems Laboratory (AESL). He received his B.S. in ME from the University of Texas at Austin and his M.S. in ME at Stanford University.}

\end{news}

\end{document}
